``Free Software is the future, future is ours'' is the motto with which
Free Software Movement Karnataka has been spreading Free Software to all
parts of society, mainly amongst engineering college faculty and
students. However our efforts were limited due to number of volunteers
who could visit different colleges physically and explain what is Free
Software and why colleges and students should use Free Software. Hence
we were looking for ways to reach out to colleges and students in a much
larger way. In the year 2013, we conducted two major Free Software camps
which were attended by close to 160 students from 25 different colleges.
But with more than 150engineering colleges in Karnataka, we still wanted
to find more avenues to reach out to studentsand take the idea of free
software in a much larger way to them.

Lab Manual running on Free Software idea was initially suggested by
Dr.~Ganesh Aithal, Head of Department, CSE, P.A. Engineering College and
Dr.~Swarnajyothi L., Principal, Jnana Vikas Institute of Technology
during our various interactions with them individually. FSMK took their
suggestions and decided to create a lab manual which will help colleges
to migrate their labs to Free Software, help faculty members to get
access to good documentation on how to conduct various labs in Free
Software and also help students by providing good and clear explanations
of various lab programs specified by the university. We were very clear
on the idea that this lab manual should be produced also from the
students and faculty members of the colleges as they knew the right way
to explain the problems to a large audience with varying level knowledge
in the subject. FSMK promotes freedom of knowledge in all respects and
hence we were also very clear that the development and release of this
lab manual should under Creative Commons License so that colleges can
adopt the manual and share,print, distribute it to their students and
there by helping us in spreading free software.

Based on this ideology, we decided to conduct a documentation workshop
for college faculty members where they all could come together and help
us produce this lab manual. As this was a first attempt for even FSMK,we
decided to conduct a mock documentation workshop for one day at Indian
Institute of Science, Bangalore on 12 Jan, 2014. Close to 40
participants attended it,mainly our students from various colleges and
we tried documenting various labs specified by VTU. Based on this
experience, we conducted a 3 day residential documentation workshop
jointly organized with Jnana Vikas Institute of Technology, Bidadi at
their campus from 23 January, 2014. It was attended by 16 faculty
members of different colleges and 40 volunteers from FSMK. The
documentation workshop was sponsored by Spoken Tutorial Project, an
initiative by Government of India to promote IT literacy through Open
Source software. Spoken Tutorials are very good learning material to
learn about various Free Software tools and hence the videos are
excellent companion to this Lab Manual. The videos themselves are
released under Creative Commons license, so students can easily download
them and share it with others. We would highly recommend our students to
go through the Spoken Tutorials while using this Lab Manual and web
links to the respective spoken tutorials are shared within the lab
manual also.

Finally, we are glad that efforts and support by close to 60 people for
around 3 months has lead to creation of this Lab Manual. However like
any Free Software project, the lab manual will go through constant
improvement and we would like the faculty members and students to send
us regular feedback on how we can improve the quality of the lab manual.
We are also interested to extend the lab manual project to cover MCA
departments and ECE departments and are looking for volunteers who can
put the effort in this direction. Please contact us if you are
interested to support us.
